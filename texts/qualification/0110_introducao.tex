
\chapter{INTRODUÇÃO}

A programação concorrente não é mais utilizada apenas para resolver problemas relacionados com a computação de alta performance. Os microprocessadores que permitem executar programas concorrentes estão disponíveis em quase todos os dispositivos. Essa nova realidade foi vislumbrada em \cite{sutter_software_2005}, e apresentada como a Revolução do Software Concorrente.

Um programa é dito concorrente quando é construído com o propósito de resolver um ou mais problemas, utilizando mais de uma linha de execução simultânea. 

Atualmente a programação concorrente é utilizada para resolver problemas corriqueiros do desenvolvimento de softwares, como por exemplo: Desacoplar interface de uma aplicação das operações mais demoradas, como as de IO ou de processamentos do back-end; Possibilitar que programas que são executados em um único nó, possa utilizar recursos de arquiteturas multi-core; Na implementação do paradigma de comunicação utilizado pela programação Web, tanto no lado do cliente quanto no lado servidor; Para auxiliar aplicações mobiles nas suas interações com serviços e dados remotos; Em soluções que se apoiam no paradigma  \textit{peer-to-peer}, seja na implementação de serviços ou na coordenação do processamento em múltiplos nós.

Apesar da programação concorrente ser objeto de estudo desde a década de sessenta \cite{1965Dijkstra}, é possível perceber que nos últimos anos a sua utilização tem sido cada vez mais incentivada, e isso pode ser notado ao avaliar a evolução das novas arquiteturas de hardware com o enfoque na disponibilização de recursos de concorrência ou com a evolução e criação de novas de linguagens de programação com o proposito de viabilizar ou facilitar a utilização desse tipo de programação.

\section{Motivação}

Para garantir o funcionamento correto de um programa que utiliza múltiplos fluxos de processamento, muitas vezes é necessário realizar algum tipo de comunicação ou sincronização das tarefas entre esses fluxos de processamento. Essa necessidade cria novos desafios de designs e de verificação da confiabilidade do programa.

Durante a criação de um programa, apenas a mudança do paradigma de programação sequencial para o de programação concorrente, já cria uma complexidade na forma de pensar o algoritmo, o que normalmente impacta na confiabilidade de um programa.

 Essa confiabilidade também é afetada pelo comportamento não determinístico na ordem de execução dos seus fluxos de processamento. O gerenciamento da ordem dos fluxos de execução de um programa concorrente é de responsabilidade exclusiva do Sistema Operacional e durante a criação do programa, o desenvolvedor não tem como interferir na definição dessa ordem. 

Nesse contexto, durante a execução de um programa concorrente, onde a execução dos fluxos de processamento acontecem de forma imprevisível, poderão surgir erros inesperados e que podem ocorrer de forma intermitente. Esse processo não determinístico dificulta o processo de identificação e reprodução desse tipo de erro. 

Algumas soluções têm sido propostas com o objetivo de aumentar a confiabilidade dos programas concorrentes, como por exemplo: A criação de linguagens de programação que tentam diminuir a complexidade da utilização desse tipo de programação; E a criação de novas técnicas e ferramentas de validação e verificação, que visam auxiliar o processo de identificação desses tipos de problemas. O interesse por ambas soluções tem  crescido tanto na comunidade acadêmica quanto na indústria de desenvolvimento de software.

\section{Caracterização do problema}

Em um programa concorrente, a necessidade de comunicação ou sincronização do trabalho entre os diversos fluxos de execução aumenta consideravelmente a complexidade desse programa. As duas principais abordagens para realizar esse tipo de comunicação são: A utilização de uma área de memória compartilhada para suportar a comunicação entre os fluxos concorrentes de trabalho que estão sendo executados em uma única máquina; Uma segunda abordagem, que pode ser utilizada em um único nó ou de forma distribuída, é que a comunicação entre esses fluxos aconteça através de troca de mensagens.

Tipicamente, a abordagem de comunicação entre fluxos mais utilizada é a de utilização de uma área compartilhada de memória. Essa abordagem é um pouco mais próxima do paradigma de programação sequencial e seria em teoria um pouco simples para quem começa a trabalhar com a programação concorrente. Entretanto existem alguns pontos comuns onde podem acontecer falhas nessa abordagem, como por exemplo quando o programador não garante um acesso exclusivo durante alterações nessa área de memória compartilhada.

 Os programas que se baseiam na abordagem de comunicação entre os fluxos através da troca de mensagem, tentaram inicialmente resolver o problema de comunicação e sincronização do trabalho para processos que são executados em máquinas diferentes. Entretanto nos últimos anos, algumas linguagens de programação como por exemplo as linguagens Erlang, Go e Rust, mesmo trabalhando com múltiplos fluxos dentro de um único nó, passaram a disponibilizar recursos que permitem que a comunicação entre esses múltiplos fluxos ocorra através da troca de mensagens, tentando evitar os problemas de concorrência que tipicamente ocorrem dentro das áreas de memórias compartilhadas.

 Ferramentas de testes e de verificação e validação de software tem se mostrado mecanismos eficientes para apoiar o desenvolvimento de aplicações. Muitos estudos, ferramentas e técnicas foram criadas para apoiar a validação de programas tradicionais \cite{myers_art_2012}, \cite{DAVIDCOWARD1988189}, \cite{umar2019study}.

 Tanto devido a complexidade inerente a programação concorrente, quanto ao aumento da utilização da programação concorrente no dia a dia do desenvolvimento de software tem demandado a criação de técnicas, ferramentas, que apoiem os processos de teste, validação e verificação de programas concorrentes. 
 
 A maioria das técnicas e ferramentas de verificação e validação existentes são voltadas para programas que realizam a comunicação entre os fluxos através de compartilhamento de memória.

 As técnicas utilizadas bem como as ferramentas de verificação e validação voltadas para programação concorrente com troca de mensagens são menores que as utilizadas com o paradigma de comunicação utilizando em memória compartilhada.

 \section{Proposta do trabalho}

Esse trabalho tem o objetivo em atuar no segundo tipo de solução, relacionadas com as técnicas, e ferramentas de validação e verificação.  

Ainda dentro desse conjunto, o trabalho tenta focar em As técnicas mais comuns de são as de memoria compartilhada, e as ferramentas são mais para memoria compartilhada.

gera demandas para estabelecer técnicas de verificação para esse tipo de problema

O que o trabalho vai fazer, que é um levantamento do que existe hoje, tanto em termos de técnicas quanto ferramentas e a identificação de assuntos pouco ou não explorados.


Dentro da estrutura arquitetural dos computadores vigentes, o sistema operacional é o responsável pelo gerenciamento das múltiplas linhas de execução de um  programa, e o desenvolvedor não possui mecanismos para interferir nesse processo de execução. 

Essa característica de imprevisibilidade da ordem de execução pode criar uma situação errática de execução, e que pode acontecer de forma aleatória. Um erro com essa tipo de característica seria de difícil detecção ou recriação, o que prejudicaria o seu processo de correção.

Mesmo com a criação de novas linguagens de programação e de novas ferramentas que auxiliam a detecção dos problemas relacionados à programação concorrente, a característica da imprevisibilidade de execução ainda torna o desenvolvimento de programas concorrentes uma atividade complexa.

Apesar de existirem esforços na tentativa de converter de  forma automática, programas escritos sob o paradigma sequencial em programas concorrentes, devido a complexidade inerente a esse tipo de programação, os esforços ainda não avançaram de forma significativa, e tipicamente esse o processo de manutenção e ajustes nesse tipo de situação, normalmente precisa ser realizado por um programador.


\section{Estrutura dessa proposta}
	o restante deste texto está organizado da seguinte forma. O Capítulo~\ref{cap:proposta_pesquisa} descreve o sistema em questão, seguido pela caracterização de usuários reais na Capítulo~\ref{cap:referencial_teorico}, e o modelo na Capítulo~\ref{cap:metodologia} com algumas fórmulas fechadas. Apresentamos simulações na Capítulo~\ref{cap:resultados_parciais}.   Finalmente, trabalhos relacionados e conclusão vêm nos Capítulos~\ref{cap:plano_cronograma}.