
% ---
% Fundamentação Teórica
% ---
\chapter{REFERENCIAL TEÓRICO}
\label{cap:referencial_teorico}
% ---

Neste capítulo, descrevemos os conceitos importantes e necessários para um maior entendimento do trabalho.

\section{Programação concorrente}

Como explicado na introdução, um programa é considerado concorrente quando tenta resolver um ou mais problemas utilizando mais de uma linha de execução simultânea. 

A programação concorrente é utilizada para resolver diversos tipos de problemas. Alguns problemas inicialmente não foi é utilizada para resolver problemas de 

Complexidade desse tipo de programacao

Quais os tipos

Precisa ser multithread ou distribuído?


\section{Programação concorrente baseada em troca de mensagem}

Tipos de comunicação entre processos

Como é a comunicação baseada em troca de mensagens

\section{Verificação e validação de programas concorrentes}

A necessidade de verificar/validar programas concorrentes

 O que é a verificação e a validação de programas

 Como podem ser realizadas?

 Verificações estáticas e dinâmicas de código.
 