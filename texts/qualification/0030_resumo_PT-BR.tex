
% resumo em português
\setlength{\absparsep}{18pt} % ajusta o espaçamento dos parágrafos do resumo
\begin{resumo}
Nos últimos anos a maneira mais eficaz para se realizar otimização no tempo de execução dos processamentos é feita através da paralelização dos processos durante a execução dos programas. O hardware dos mais variados tipos de dispositivos que possuem poder computacional atualmente, como micro-computadores ou dispositivos móveis, habitual mente já possuem arquitetura que permite a realização dos seus processamentos de forma paralela.
Nesse contexto, o desenvolvimento de programas que podem ser executados de forma paralela tem sido cada vez mais necessário e utilizado por uma maior quantidade de desenvolvedores. fórmulas fechadas que  aproximam a probabilidade de infecção e contrastamos os \emph{insights} do modelo com simulações. Simulações suportam qualitativamente as observações do modelo epidêmico e estendem a análise permitindo distribuições gerais.

 \textbf{Palavras-chave}: programação paralela, problemas de concorrência, verificação de erros, SLR, revisão de literatura sistemática.
\end{resumo}