
% ---
% Evidência no mundo real
% ---
\chapter{PROPOSTA DE PESQUISA }
\label{cap:proposta_pesquisa}
% ---

A proposta dessa pesquisa visa tentar entender qual o estado da arte e da prática nas pesquisas relacionadas com a verificação, validação e testes de programas concorrentes que se baseiam em troca de mensagens, bem como quais as lacunas existentes nessa área de pesquisa.

Todo o levantamento e análise que será realizado nessa pesquisa vai se apoiar na metodologia de revisão sistemática de literatura. Maiores detalhes sobre a metodologia e como ela será utilizada nesse trabalho é apresentada no capítulo \ref{cap:metodologia} desse trabalho.

 
\section{Objetivo principal} \label{sec:objetivos}
Identificar qual o estado atual das pesquisas relacionadas com as técnicas de verificação, validação e testes de programas concorrentes que se baseiam em troca de mensagens, bem como as lacunas existentes atualmente nos estados da arte e da prática.
	
\section{Objetivos específicos} 

Além dos questionamentos estabelecidos no objetivo principal desse trabalho, como parte dos objetivos específicos, algumas outras contribuições para essa área de pesquisa e algumas perguntas serão exploradas, além de e dentre elas podemos destacar: 

\begin{enumerate}
    \item Identificar de uma forma sistemática quais os principais problemas de concorrência existentes nos programas que se baseiam em trocas de mensagem?
    \item Entender como esses problemas são verificados e validados?
    \item Identificar quais as principais técnicas utilizadas nesse processo de validação e verificação de programas concorrentes que se baseiam em troca de mensagens?
    \item Identificar quais as ferramentas de testes existem hoje com foco em programas concorrentes que se apoiam em troca de mensagens?
    \item Tentar entender se os problemas que ocorrem tipicamente em programas concorrentes que se baseiam na utilização de memória compartilhada, também ocorrem com a troca de mensagens.
    \item Criar uma revisão sistemática de literatura que possa ser usada como ponto de partida para o trabalhos de outros pesquisadores que já atuem ou queiram atuar nessa área de pesquisa
\end{enumerate}

\section{Contribuições esperadas} 

As principais contribuições esperadas como resultado desse trabalho são:

\begin{enumerate}
    \item Produzir um trabalho que consolide o conhecimento sobre o assunto de técnicas de validação e verificação de programas concorrentes que se baseiam em troca de mensagens. Durante o levantamento inicial, não foram encontrados trabalhos específicos sobre esse tema.
    \item Que devido a metodologia escolhida ser um processo metódico, repetível e minimizando a possibilidade de enviesamento, o mesmo possa ser utilizado de uma maneira mais consistente como um ponto de partida para o trabalhos de outros pesquisadores que já atuem ou queiram atuar nessa área de pesquisa.
\end{enumerate}