
% ---
% Modelo Epidemiológico
% ---
\chapter{METODOLOGIA}
\label{cap:metodologia}
% ---
   A metodologia que será utilizada nesse trabalho será a Revisão Sistemática de Literatura. 
   
   A ideia por trás da escolha de desse metodologia, é tornar todo o processo da pesquisa mais metódico,  repetível e tentando também reduzir a possibilidade de enviesamento do estudo.  
	
\section{Revisão Sistemática de Literatura}

O trabalho de \cite{keele_guidelines_2007} apresenta alguns pontos interessantes sobre a metodologia da Revisão Sistemática de literatura. Dentre eles os seguintes pontos chamam mais atenção:

\begin{enumerate}
    \item A revisão sistemática reúne de forma rigorosa a literatura de todas as pesquisas relacionadas a um problema específico, fornecendo conhecimento sobre esse problema;
    \item A metodologia bem definida diminui a possibilidade de enviesamento (apesar de não eliminar um possível enviesamento do estudo selecionado);
    \item Pode prover informações sobre um fenômeno, e a depender do resultado criar evidências se é consistente ou se as variações precisam ser melhor estudadas;  
    \item Avalia criticamente as pesquisas;
    \item Identificar em áreas de pesquisa com intuito de sugerir futuras investigações;
    \item Para prover um framework que pode de maneira apropriada posicionar novas atividades de pesquisa;
    \item Tem origem nos estudos voltados para a área de saúde;
    \item Tem sido utilizada em algumas áreas de Sistema de Informação, principalmente na Engenharia de Software.
    \item Requer muito mais esforço que uma revisão tradicional.
\end{enumerate}

Segundo \cite{Cordeiro-2007}, as primeiras técnicas formais da combinação dos resultados de diferentes estudos foram elaboradas e publicadas no British Medical Journal, pelo matemático britânico Karl Pearson, em 1904. Em 1955, aparece a primeira revisão sistemática sobre um cenário clínico, publicada no Journal of American Medical Association.

Também na década de 90, apareceram as duas primeiras teses que consistiam em revisões sistemáticas com metanálises: uma na Inglaterra e outra no Brasil.

Em 1995, um grupo de cientistas reunidos em Potsdam (Alemanha) definiu como revisão sistemática “a aplicação das estratégias científicas que limitam o viés de seleção e avaliam com espírito científico os artigos e sintetizam todos os estudos relevantes em tópicos específicos.

De acordo com \cite{Dyba2005}, outras áreas do conhecimento humano passaram a utilizar metodologias chamadas de Baseadas em evidências, como a Enfermagem, Cuidados primários, Química Orgânica, Psicologia Empírica e Educação. A engenharia de software demonstrou ter similaridade com as áreas de ciências sociais. Com isso alguns estudos tem sido realizados utilizando a metodologia.

\section{Plano da Revisão Sistemática} \label{sec:planoRevisao}

 Uma das etapas iniciais de um trabalho que utiliza a metodologia de revisão sistemática de literatura é a criação de um plano de revisão de literatura. Esse plano é tipicamente criado  antes da execução da revisão. O plano contém as definições de etapas, formato e regras que serão utilizadas no trabalho. O plano de revisão utilizado nesse trabalho é detalhado nessa seção e também se baseou no exemplo do trabalho \cite{kitchenham2007guidelines}, que é um guia sobre como aplicar a metodologia em trabalhos relacionados com o assunto Engenharia de Software. Outras ideias foram adicionadas com base no trabalho desenvolvido em \cite{de2023gestao}.

\subsection{Background}
Apesar das citações de trabalhos de sistemas da informação que se apoiaram na metodologia de revisão sistemática de literatura apresentar um crescimento durante o passar dos anos, a quantidade de trabalhos retornados durante buscas por revisões sistemáticas de literatura nessa área ainda é pouco expressivo, como demonstrado em \cite{kitchenham_systematic_2009}.

Apesar de existir perceber um esforço na realização de trabalhos nesse formato em algumas áreas da informática, como Sistema da Informação e Engenharia de Software, muitas áreas ainda não tem essa prática.

Ao tentar levantar revisões sistemáticas voltadas para as áreas de validação, verificação, técnicas e testes automatizados de programação concorrente, como apresentado no capítulo  \ref{cap:trabalhos_relacionados}, foram encontrados apenas três trabalhos e todos com mais de dez anos de publicação. Ao tentar restringir mais o escopo da busca, para o tema de interesse desse trabalho, tentando focar nos assuntos de validação, verificação, técnicas e testes automatizados de programação concorrente baseados em troca de mensagem, nenhum resultado foi encontrado e esse é mais uma das motivações para a utilização dessa metodologia.

\subsection{Perguntas de Pesquisa} \label{subsec:pergutasPesquisa}

As perguntas que devem pretendem ser respondidas ao final dessa revisão sistemática de literatura são apresentadas a seguir:

\begin{enumerate}
    \item Quais problemas de concorrência aparecem em programas concorrentes baseados em trocas de mensagens?
    \item Quais técnicas têm sido desenvolvidas para identificar esses problemas?
    \item Como programas concorrentes baseados em troca de mensagem são verificados e validados?
    \item Dos problemas de concorrência que tipicamente ocorrem em programas baseados em memória compartilhada, como: corrida de dados; violação de ordem; violação de atomicidade e deadlocks, quais também ocorrem no contexto da troca de mensagens?
    \item Qual os lacunas existentes nessa área de conhecimento, atualmente
\end{enumerate}

\subsection{Fluxo de trabalho} \label{subsec:fluxoTrabalho}

O processo será feito seguindo um fluxo tradicional de um trabalho de SLR com as seguintes fases:

\begin{enumerate}
    \item Planejamento
    \begin{enumerate}
        \item Desenvolver Protocolo da Revisão de Literatura
            \begin{enumerate}
                \item Objetivo da pesquisa
                \item Questões de pesquisa
                \item Palavras chaves e string de busca
                \item Bases que serão pesquisadas
                \item Critérios de inclusão e exclusão
                \item Formulário de qualidade
                \item Formulário de extração de dados
            \end{enumerate}
        \item Avaliar/Revisar Protocolo da Revisão de Literatura
    \end{enumerate}
    \item Execução
    \begin{enumerate}
        \item Identificar SLRs e Surveys relevantes
        \item Identificar estudos primários relevantes 
        \item Selecionar estudos primários
        \item Verificar qualidade dos estudos (se atende aos critérios de qualidade estabelecidos no plano)
        \item Realizar extração de dados com formulário pré definido no plano
        \item Sintetizar dados levantados
        \item Avaliar/Revisar dados sintetizados
    \end{enumerate}
    \item Reportar 
    \begin{enumerate}
        \item Escrever relatório / artigo da revisão
        \item Validar relatório de revisão
    \end{enumerate}
\end{enumerate}

Uma diferença substancial sobre o fluxo proposto, é que a ideia planejada para esse trabalho é a de realizar todas as etapas de forma iterativa. Cada iteração deverá pesquisa sobre uma das bases de dados escolhidas (começando da mais importante para as menos importantes). Espera-se que com essa abordagem, seja possível melhorar o processo, fazer ajustes e possibilite que a produção do texto do trabalho possa ser realizada de forma incremental.

\subsection{String de busca}

A string de busca que será utilizada no trabalho obedecerá a seguinte estrutura:


\begin{verbatim}
(
    ("title + abstract":"message passing")  
    AND(
              "title + abstract":parallel*
              OR "title + abstract":distribut*
              OR "title + abstract":multi-thread*
              OR "title + abstract":concurren* ) 
    AND (
              "title + abstract":test* 
              OR "title + abstract":check*
              OR "title + abstract":validat* 
              OR "title + abstract":verif*
              OR "title + abstract":correct*
              OR "title + abstract":techni*
             )
)    
\end{verbatim}

\subsection{Databases que serão pesquisados}  \label{subsec:repositoriosPesquisa}

O Processo de busca será feito utilizando palavras chaves nas bases de dados listadas na tabela que segue:

\begin{center}
\begin{tabular}{ |p{4cm}|p{7cm}|  }
\hline
\textbf{Origem} & \textbf{Responsável pela execução} \\
\hline
IEEE & Saulo Andrade Almeida \\ 
\hline
ACM & Saulo Andrade Almeida \\ 
\hline
Science Direct & Saulo Andrade Almeida \\ 
\hline
Springer Link & Saulo Andrade Almeida \\ 
\hline
SCOPUS & Saulo Andrade Almeida \\ 
\hline
Citeseer library & Saulo Andrade Almeida \\ 
\hline
SciElo & Saulo Andrade Almeida \\ 
\hline
arXiv & Saulo Andrade Almeida \\ 
\hline
\end{tabular}
\end{center}


\subsection{Critérios de Inclusão}

Os critérios de inclusão definidos para a escolha de trabalhos serão:

\begin{enumerate}[label=CI\arabic*:]
    \item Systematic Literature Reviews (SLRs) sobre: Validação; Verificação, Técnicas e Testes automatizados de programação concorrente baseados em troca de mensagem.
    \item Literature surveys sobre os mesmo assuntos listados no primeiro item.
    \item Ter relação com algum dos sobre os mesmo assuntos listados no primeiro item
\end{enumerate}


\subsection{Critérios de Exclusão}

As seguintes regras serão utilizadas no processo de exclusão de artigos utilizados no trabalho:

\begin{enumerate}[label=CE\arabic*:]
    \item Artigos em idioma diferentes do inglês e português;
    \item Artigo duplicado;
    \item Artigos que não tenham relação com o tema definido nos critérios de Inclusão
    \item Artigos primários que não possam ajudar a responder às questões de pesquisa definidas
    \item SLRs e Surveys que não possuam questões de pesquisa, sem processos de busca definido, Sem definição de extração e análise de dados). 
    \item Quando um mesmo trabalho for apresentado em mais de uma Jornal/Conferência, a versão mais completa será utilizada.
    \item Quando não for possível obter uma cópia do artigo.
\end{enumerate}

\subsection{Processo de seleção dos estudos}

O processo de seleção dos estudos primários será dividido em cinco etapas. A seleção será feita por um pesquisador a revisão será realizada por outro. As etapas do processo de seleção dos estudos se dará da seguinte forma:

\begin{enumerate}[label=Etapa\arabic*:]
    \item Nessa primeira etapa, a string de busca será aplicada na base de pesquisa da iteração e dos resultados retornados serão removidos os trabalhos que estejam em idioma diferentes dos que foram definidos no critério de exclusão CE1 e CE2;
    \item Na segunda etapa, com base no título, resumo e palavras chaves, remover artigos considerados irrelevantes para a pesquisa, de acordo com os critérios de exclusão CE3;
    \item Na terceira etapa, agora também com base na introdução e conclusão, remover artigos considerados irrelevantes para a pesquisa (referente aos critérios de aceitação CE3, CE4, CE5) e artigos que não tiver acesso (referente ao critério CE7);
    \item Na quarta etapa, será realizada a leitura completa dos artigos, seguido da remoção dos  artigos considerados irrelevantes para a pesquisa, seguindo os critérios de aceitação CE3, CE4, CE5 e CE6;
    \item Na quinta etapa, os estudos rejeitados serão revisados através de um processo de amostragem por outro pesquisador. 
\end{enumerate}

Será mantida uma lista de estudos candidatos e rejeitados, bem como, qual o motivo da rejeição. 

\subsection{Quality Assessment}

A verificação de qualidade para os SLRs/Surveys e para os estudos primários selecionados respeitando os seguintes critérios: 

\subsubsection{SLR e Surveys}

Pretende utilizar critérios estabelecidos pela Centre for Reviews and Dissemination \cite{crc2023}  da York University \cite{york2023}, que se apoia basicamente em  quatro questões:

\begin{enumerate}
    \item Os critérios de inclusão e exclusão foram descritos de forma apropriada?? 
    \item A revisão de literatura aparentemente cobriu todos os estudos relevantes? 
    \item Os revisores avaliaram a qualidade/validade dos estudos que foram incluídos? 
    \item Os estudos e dados básicos foram adequadamente descritos?
\end{enumerate}

As questões serão pontuadas da seguinte forma:

\begin{enumerate}[label=Questão \arabic*:]
\item S (Sim), os critérios de inclusão são explicitamente definidos no artigo, P (Parcialmente), os critérios de inclusão são implícitos; N (Não), os critérios de inclusão não estão definidos e não podem ser inferidos. 
\item S, os autores pesquisaram 4 ou mais bibliotecas digitais e incluíram estratégias de pesquisa adicionais ou identificaram e referenciaram todos os periódicos que abordam o tópico de interesse; P, os autores pesquisaram 3 ou 4 bibliotecas digitais sem estratégias de pesquisa extras, ou pesquisaram um conjunto definido, mas restrito de periódicos e anais de conferências; N, os autores pesquisaram até 2 bibliotecas digitais ou um conjunto extremamente restrito de periódicos. 
\item S, os autores definiram explicitamente critérios de qualidade e os extraíram de cada estudo primário; P, a questão de pesquisa envolve questões de qualidade que são abordadas pelo estudo; N, nenhuma avaliação explícita da qualidade de artigos individuais foi tentada.
\item S, São apresentadas informações sobre cada artigo; P, apenas informações resumidas são apresentadas sobre artigos individuais; N, os resultados dos estudos individuais não são especificados.
\end{enumerate}

O Procedimento de pontuação é a seguinte: S=1, P=0.5 e N ou Desconhecido=0. 

O dado será extraído por um pesquisador e uma amostra randômica de 10\% do tamanho dos dados será verificada por outro. 

\subsubsection{Estudos primários}

As questões dos estudos primários serão pontuadas da seguinte forma:

\begin{enumerate}[label=Questão \arabic*:]
\item S (Sim), O artigo consegue responder a todas as perguntas de pesquisa? P (Parcialmente), o artigo consegue responder ao menos duas das perguntas de pesquisa; N (Não), o artigo consegue responder uma pergunta de pesquisa. 
\item S, O experimento do artigo consegue ser reproduzido de maneira fácil e clara; P, o experimento do artigo está disponível, mas não foi possível reproduzir; N, não foi possível encontrar ou reproduzir o experimento do artigo.
\end{enumerate}

O Procedimento de pontuação é a seguinte: S=1, P=0.5 e N ou Desconhecido=0.

O dado será extraído por um pesquisador e uma amostra randômica de 10\% do tamanho dos dados será verificada por outro. 

\subsection{Coleta dos dados}

Os dados coletados em cada artigo serão:

\begin{enumerate}
    \item Origem (i.e. Conferência ou Jornal). 
    \item O ano que foi publicado. Nota: Se o artigo foi publicado em diversas origens, ambas as datas serão registradas, e a primeira data será utilizado na análise. 
    \item Classificação do artigo 
    \begin{enumerate}
        \item Tipo (Systematic Literature Review SLR, Survey, Estudo Primário). 
    \end{enumerate}
    \item Área principal do tópico. 
    \item Autor(es) e afiliação (organização e país) 
    \item Qual a base de pesquisa que o trabalho foi encontrado
    \item O artigo fala sobre problemas de concorrência encontrados por programas que utilizam troca de mensagens? Quais foram eles?
    \item Entre os problemas listados no trabalho, existem problemas tipicamente na programação concorrente baseado em memória compartilhada? Quais foram eles?
    \item O artigo fala sobre validação de programas concorrentes que utilizam troca de mensagens? Quais foram as técnicas utilizadas?
    \item O artigo fala sobre verificação de programas concorrentes que utilizam troca de mensagens? Quais foram as técnicas utilizadas?
    \item O artigo propõe ou utiliza alguma ferramenta? Quais? Estão acessíveis?    
    \item Quais os Gaps existentes no estudo.
    \item Resumo do artigo. 
    \item Pontuação do estudo. 
\end{enumerate}

Os dados serão extraídos por um pesquisador e uma amostra randômica de 10\% do tamanho dos dados será verificada por outro. 

\subsection{Análise dos dados}

Os dados serão tabulados (ordenados alfabeticamente pelo primeiro nome do autor) para mostrar a informação básica de cada estudo. O número de estudos em cada categoria principal será contada.

As tabelas serão revisadas para responder às questões de pesquisa e identificar quaisquer tendências ou limitações interessantes na pesquisa atual relacionada, como segue:

\begin{enumerate}[label=Pergunta \arabic*:]
    \item Quais os principais problemas encontrados em programas concorrentes baseados em troca de mensagens? Isso será abordado identificando e totalizando as respostas dos formulários de extração de dados.
    \item Quais as técnicas de pesquisa que estão sendo abordadas nas pesquisas?  Isso será abordado identificando e totalizando as respostas dos formulários de extração de dados.
    \item Como os programas concorrentes são verificados e validados?  Isso será abordado identificando e totalizando as respostas dos formulários de extração de dados.
    \item Quais os problemas que ocorrem em programas concorrentes que se baseiam em memória compartilhada e que também ocorrem com os programas que se baseiam em troca de mensagem?  Isso será abordado identificando e totalizando as respostas dos formulários de extração de dados.
    \item Quais são as limitações da pesquisa atual? Analisaremos a variedade de tópicos, o escopo e a qualidade para determinar se existem limitações observáveis.
\end{enumerate}

Além da análise também serão tabulados os seguinte resultados:

\begin{enumerate}
    \item Número de artigos selecionados por ano e por origem 
    \item Distribuição dos artigos selecionados por base de pesquisa
\end{enumerate}


\subsection{Disseminação}

Os resultados do estudo devem ser de interesse das comunidades que possuem como áreas de interesse os seguinte assuntos: de programação concorrente; sistemas distribuídos; troca de mensagem; validação e verificação de software e possivelmente testes automatizados. 

Por esse motivo, planejamos relatar os resultados no seguinte repositório do github \cite{repositorioGithub}. 

Também será documentado o resultado completo do estudo na dissertação de mestrado do Programa de Pós Graduação em Informática da Universidade Federal do Rio de Janeiro. 

Uma versão curta do trabalho será submetida para alguma conferência ou jornal relevante para a área de estudo do trabalho. Os principais congressos e jornals selecionados como candidatos para a publicação desse trabalho são:


\begin{center}
\begin{tabular}{|p{8cm}|p{2cm}|p{3cm}|p{2cm}|}
\hline
\textbf{Nome} & \textbf{Dt último evento} & \textbf{Dt limite submissão} & \textbf{Tipo publicação} \\
\hline 
International Workshop on Software Correctness for HPC Applications & Nov/22 & Ago/22 & Conferência \\ 
\hline 
International Conference on Automated Software Engineering & Set/22 & Abr/22 & Conferência \\ 
\hline 
IEEE Access & N/A & Qualquer data & Jornal \\ 
\hline 
IEEE TRANSACTIONS ON PARALLEL AND DISTRIBUTED SYSTEMS & N/A & Qualquer data & Jornal \\ 
\hline 
IEEE TRANSACTIONS ON SOFTWARE ENGINEERING & N/A & Qualquer data & Jornal \\ 
\hline 
ACM COMPUTING SURVEYS & N/A & Qualquer data & Jornal \\ 
\hline 
ACM TRANSACTIONS ON PROGRAMMING LANGUAGES AND SYSTEMS & N/A & Qualquer data & Jornal \\ 
\hline 
CONCURRENCY AND COMPUTATION: PRACTICE \& EXPERIENCE & N/A & Qualquer data & Jornal \\ 
\hline 
JOURNAL OF PARALLEL AND DISTRIBUTED COMPUTING & N/A & Qualquer data & Jornal \\ 
\hline 
THE JOURNAL OF SYSTEMS AND SOFTWARE & N/A & Qualquer data & Jornal \\ 
\hline
\end{tabular}
\end{center}