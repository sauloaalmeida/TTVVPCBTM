
% ---
% Trabalhos relacionados
% ---
\chapter{TRABALHOS RELACIONADOS}
\label{cap:trabalhos_relacionados}
% ---


Neste capítulo, listamos trabalhos relacionados ao assunto de testes e técnicas de validação e verificação de programas concorrentes. Nessa seção foram levantados principalmente trabalhos de revisão sistemática de leitura, surveys voltados para o estado da arte e da prática. Durante essa etapa de exploração, a busca não foi limitada apenas para programação concorrente que se baseia em troca de mensagem.

Em 1992, o artigo \cite{taylor_structural_1992} estende a ideia de critérios de testes estruturais e propõe técnicas de testes que oferecem suporte à testes estruturais voltados para programação concorrente. Os critérios de cobertura descritos incluem: cobertura de estado, cobertura de transição do estado e cobertura de sincronização.  Também estão incluídos um analisador estático de concorrência e um monitor em tempo de \textit{runtime}. As técnicas foram propostos para Ada e CSP. Os melhores resultados foram obtidos com programas que tinham apenas nomeação estática de tarefas de objetos

Em 2001, o autor do artigo \cite{al-iadan_survey_2001} apresenta um \textit{survey} e uma taxonomia das abordagens existentes para a detecção de erros de condição de corrida e \textit{deadlocks} em programas dos tipos paralelos e distribuídos, apresentando uma breve discussão que destaca os principais problemas nas abordagens mais conhecidas. 

Já em 2006, no artigo \cite{wojcicki_state--practice_2006}, através de um questionário, foi realizando um levantamento do estado da prática com trinta e cinco participantes, sobre pesquisas em Verificação e Validação para programas concorrentes.

O artigo \cite{mamun_concurrent_2009} de 2009, realizou uma revisão sistemática para explorar o estado da arte de testes de software concorrente, e o relatório dessa revisão apresentou características de software concorrente, bugs, levantamento de técnicas de testes, de ferramentas, \textit{benchmarks} desenvolvidos para ferramentas. O trabalho também apresenta uma avaliação de quatro ferramentas proprietárias e \textit{open-sources} de análise estática de software para detecção de erros de programação \textit{multithread} em Java.

Em 2010, no artigo \cite{brito_concurrent_2010} foi realizada uma revisão sistemática de literatura com o objetivo de entender o contexto da pesquisa daquele momento, e como resultado a revisão identificou critérios de testes, taxonomia de bugs, e ferramentas de testes na área de testes de programação concorrente. 

No artigo \cite{abdelqawy_survey_2012} de 2012, o \textit{survey} identifica ferramentas e técnicas de teste e depuração de programas concorrentes e \textit{multi-thread}, destacando as diferentes formas de implementação dessas técnicas.

Em 2016, no artigo \cite{arora_systematic_2016}, também foi feita uma revisão sistemática de literatura e como resultado foi criada uma classificação de oito categorias de técnicas e abordagens utilizadas em testes de programas concorrentes. 

No \textit{survey} \cite{lima_survey_2017} de 2017, foi realizada uma análise sobre Testes em Sistemas que rodam em dispositivos móveis, plataformas de nuvem e sobre mais de um domínio. Esse ambiente heterogênio tipicamente conhecido pelo nome  \textit{Distributed and heterogeneous systems} (DHS). A pesquisa buscou avaliar o estado da prática, entrevistando cento e quarenta e sete profissionais de testes participantes de conferências relacionadas com a industria de testes de software.

O último survey encontrado \cite{bianchi_survey_2018} em 2018, apresenta um levantamento bem vasto, sobre tendencias recentes em testes de sistemas de software concorrentes. Propõe uma classificação que visa ampliar o entendimento de pontos fracos e fortes de cada abordagem, bem como empecilhos para os avanços da pesquisa. O estudo sinaliza que existem lacunas relacionadas com testes, verificação e validações de códigos baseados em troca de mensagem. 

Não foi encontrado trabalhos dedicados exclusivamente para testes, verificação e validação de programação concorrente que se apoiam em troco de mensagens.

Entende-se que esse levantamento inicial muito provavelmente será ampliando durante o levantamento bibliográfico que faz parte do processo da revisão sistemática.